\secnumberlesssection{INTRODUCCIÓN}

%Debe proporcionar a un lector los antecedentes suficientes para poder contextualizar en general la situación tratada, a través de una descripción breve del área de trabajo y del tema particular abordado, siendo bueno especificar la naturaleza y alcance del problema; así como describir el tipo de propuesta de solución que se realiza, esbozar la metodología a ser empleada e introducir a la estructura del documento mismo de la memoria.

%En el fondo, que el lector al leer la Introducción pueda tener una síntesis de cómo fue desarrollada la memoria, a diferencia del Resumen dónde se explicita más qué se hizo, no cómo se hizo.

%El ser humano es un investigador innato, y desde sus orígenes ha buscado y estudiado la materia que lo rodea y cómo funciona. A partir de ello y a lo largo de los años, nacieron una gran variedad de campos de estudio, entre ellos, la física de partículas o también llamada \emph{física de alta energía}, la cual se dedica al estudio de las partículas que constituyen la materia y la radiación. 



% Qué poner aquí:
% Por qué es necesario colisionar protones o electrónes a altas energías para estudiar las dimensiones más pequeñas de la materia?
% Cuál es la importancia del estudio de física de partículas, bosón de Higgs.

